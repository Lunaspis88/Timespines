\documentclass[12pt,letterpaper]{article}
\usepackage{natbib}

%Packages
\usepackage{pdflscape}
\usepackage{fixltx2e}
\usepackage{textcomp}
\usepackage{fullpage}
\usepackage{float}
\usepackage{latexsym}
\usepackage{url}
\usepackage{epsfig}
\usepackage{graphicx}
\usepackage{amssymb}
\usepackage{amsmath}
\usepackage{bm}
\usepackage{array}
\usepackage[version=3]{mhchem}
\usepackage{ifthen}
\usepackage{caption}
\usepackage{hyperref}
\usepackage{amsthm}
\usepackage{amstext}
\usepackage{enumerate}
\usepackage[osf]{mathpazo}
\usepackage{dcolumn}
\usepackage{lineno}
\usepackage{dcolumn}
\newcolumntype{d}[1]{D{.}{.}{#1}}

\pagenumbering{arabic}


%Pagination style and stuff
\linespread{2}
\raggedright
\setlength{\parindent}{0.5in}
\setcounter{secnumdepth}{0} 
\renewcommand{\section}[1]{%
\bigskip
\begin{center}
\begin{Large}
\normalfont\scshape #1
\medskip
\end{Large}
\end{center}}
\renewcommand{\subsection}[1]{%
\bigskip
\begin{center}
\begin{large}
\normalfont\itshape #1
\end{large}
\end{center}}
\renewcommand{\subsubsection}[1]{%
\vspace{2ex}
\noindent
\textit{#1.}---}
\renewcommand{\tableofcontents}{}
%\bibpunct{(}{)}{;}{a}{}{,}

%---------------------------------------------
%
%       START
%
%---------------------------------------------

\begin{document}

\begin{abstract}
Body armour is a consistent trait observed in metazoans ecosystems throughout the whole phanerozoic fossil record.
This is due, in part because of it's taphonomy (i.e. being relatively well preserved c.f. soft tissue characters) but also to it's ecological importance.
In fact, morphological diversity have been hypothesised to rise at a really fast rate in part due to the role of body armour in the predator prey evolutionary arms race \citep{GouldWonderful}.
Under this hypothesis, we expect the evolution of body armour in metazoans to be tightly correlated to predator's morphological evolution.
More precisely, we expect that preys that are too small or to big relatively to an ecosystem's dominant predator will be less likely to evolve body armour.
Conversely, preys that fall in the ``danger zone'', (i.e. falling in the size range of their dominant predator) will be more likely to evolve body armour.

Recent studies have demonstrated the presence of such `danger zone'' in extant mammals where medium sized ones where more likely to have specialised defence mechanisms (including body armour) than smaller or bigger ones \citep{Stankowich2016}.
The aim of this project is thus to expend this study and investigate whether this ``danger zone'' existed for all vertebrates throughout the phanerozoic and whether it was a driver of morphological evolution.
\end{abstract}

% \newpage

% \subsection{Things to take into account:}
% \begin{itemize}
%     \item Which clade do we consider? We need to use traits we can consider homologuous. Also we need phylogenies!
%     \item How do we measure body armour? E.g. how do we count the different between a Gliptodon or a placoderm?
%     \item Which length of time do we consider? I.e. how do we deal with the non-continuous fossil record?
% \end{itemize}

\bibliographystyle{sysbio}
\bibliography{References}

\end{document}